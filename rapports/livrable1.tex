\documentclass[a4paper]{report}

\usepackage[utf8]{inputenc}
\usepackage[T1]{fontenc}
\usepackage[francais]{babel}


\title{Rubik'INT \\ Livrable 1}
\author{Rémy \bsc{ZIRNHELD} \and Alban \bsc{MANZANO} \and Florian \bsc{GRANTE} \and Nicolas {BONNET}}
\date{7 février 2017}

\begin{document}

\maketitle

\tableofcontents

\chapter*{Introduction}
\addcontentsline{toc}{chapter}{Introduction}
Le projet de développement informatique nous a offert l'opportunité de créer le programme que nous voulions.
Divers sujets se présentaient à nous. Nous étions déterminés à trouver un projet nous permettant de mettre en oeuvre plusieurs domaines de l'informatique. 
Nous avons choisi le Rubik's Cube, un casse-tête fascinant, un cube constitué de 3x3x3 sous-cubes colorés pouvant tourner selon trois axes. 
Le Rubik's Cube se distingue par l'immense complexité de sa résolution contrastant avec sa simplicité apparente.
Chaque personne capable de résoudre le cube doit suivre une méthodologie stricte et faire preuve d'une certaine intuition.
L'objectif derrière ce célèbre casse tête est d'apprendre des domaines de l'informatique qui sont nouveau pour nous.

Le défi que nous nous sommes lancé est d'apprendre à une machine à résoudre ce casse-tête.
Pour cela nous devrons nous confronter à plusieurs problématiques.
La première, la plus évidente, est sa résolution. Nous allons avoir besoin de faire appel à des connaissances mathématiques et algorithmiques.
De plus, nous voulons donner un aspect pratique à notre projet. Ainsi l'utilisateur devra pouvoir réussir à résoudre son cube grâce au programme.
Il est donc indispensable d'élaborer une interface utilisateur.
De plus, nus devrons aussi élaborer une solution simple pour reconnaitre la configuration du cube dans le programme. Nous utiliserons un algorithme de \textit{computer vision} sur un flux vidéo acquis en temps réel par une webcam.
Ainsi ce projet regroupe les thématiques du traîtement d'image, de l'algorithmie et du traîtement d'image.


\chapter{Cahier des charges}
Cdcf

\chapter{Interface}
Interface

\chapter{Structure du code}
Structure

\chapter*{Conclusion}
\addcontentsline{toc}{chapter}{Conclusion}

\end{document}
