\documentclass[a4paper]{report}

%====================== PACKAGES ======================

\usepackage[french]{babel}
\usepackage[utf8x]{inputenc}
%pour gérer les positionnement d'images
\usepackage{float}
\usepackage{amsmath}
\usepackage{graphicx}
\usepackage[colorinlistoftodos]{todonotes}
\usepackage{url}
%pour les informations sur un document compilé en PDF et les liens externes / internes
\usepackage{hyperref}
%pour la mise en page des tableaux
\usepackage{array}
\usepackage{tabularx}
%pour utiliser \floatbarrier
%\usepackage{placeins}
%\usepackage{floatrow}
%espacement entre les lignes
\usepackage{setspace}
%modifier la mise en page de l'abstract
\usepackage{abstract}
%police et mise en page (marges) du document
\usepackage[T1]{fontenc}
\usepackage[top=2cm, bottom=2cm, left=2cm, right=2cm]{geometry}
%Pour les galerie d'images
\usepackage{subfig}
%multirow table
\usepackage{multirow}
%Source code
\usepackage{listings}

\usepackage{color}
\usepackage{hhline}

\definecolor{pblue}{rgb}{0.13,0.13,1}
\definecolor{pgreen}{rgb}{0,0.5,0}
\definecolor{pred}{rgb}{0.9,0,0}
\definecolor{pgrey}{rgb}{0.46,0.45,0.48}

\usepackage{listings}
\lstset{language=Java,
  showspaces=false,
  showtabs=false,
  breaklines=true,
  showstringspaces=false,
  breakatwhitespace=true,
  commentstyle=\color{pgreen},
  keywordstyle=\color{pblue},
  stringstyle=\color{pred},
  basicstyle=\ttfamily,
  moredelim=[il][\textcolor{pgrey}]{$$},
  moredelim=[is][\textcolor{pgrey}]{\%\%}{\%\%}
}

%====================== INFORMATION ET REGLES ======================

%rajouter les numérotation pour les \paragraphe et \subparagraphe
\setcounter{secnumdepth}{4}
\setcounter{tocdepth}{4}

\hypersetup{							% Information sur le document
pdfauthor = {Rémy ZIRNHELD,
			Alban MANZANO,
			Florian GRANTE,
    		Nicolas BONNET},			% Auteurs
pdftitle = {Rubik'INT -
			Solveur de Rubik's cube},			% Titre du document
pdfsubject = {Livrable 1},		% Sujet
pdfkeywords = {rubikscube, tsp, informatique},	% Mots-clefs
pdfstartview={FitH}}					% ajuste la page à la largueur de l'écran

%======================== DEBUT DU DOCUMENT ========================

\begin{document}

%régler l'espacement entre les lignes
\newcommand{\HRule}{\rule{\linewidth}{0.5mm}}

%page de garde
\begin{titlepage}
\begin{center}

% Upper part of the page. The '~' is needed because only works if a paragraph has started.
\includegraphics[width=0.35\textwidth]{./logo}~\\[1cm]

\textsc{\LARGE PRO3600 - Projet informatique}\\[1.5cm]

\textsc{\Large }\\[0.5cm]

% Title
\HRule \\[0.4cm]

{\huge \bfseries Rubik'INT\\
Solveur de Rubik's Cube \\[0.4cm] }

\HRule \\[1.5cm]

% Author and supervisor
\begin{minipage}{0.4\textwidth}
\begin{flushleft} \large
\emph{Auteurs:}\\
Rémy \textsc{ZIRNHELD}\\
Alban \textsc{MANZANO}\\
Florian \textsc{GRANTE}\\
Nicolas \textsc{BONNET}
\end{flushleft}
\end{minipage}
\begin{minipage}{0.4\textwidth}
\begin{flushright} \large
\emph{Tutrice:} \\
Amina \textsc{GUERMOUCHE}
\end{flushright}
\end{minipage}

\vfill

% Bottom of the page
{\large \today}

\end{center}
\end{titlepage}


%ne pas numéroter cette page
\thispagestyle{empty}
%\newpage

%%%\input{./abstract.tex}

\tableofcontents
\thispagestyle{empty}
\setcounter{page}{0}
%ne pas numéroter le sommaire

%\newpage

%espacement entre les lignes d'un tableau
\renewcommand{\arraystretch}{1.5}

%====================== INCLUSION DES PARTIES ======================

~
\thispagestyle{empty}
%recommencer la numérotation des pages à "1"
\setcounter{page}{0}
\newpage

\chapter*{Introduction}
\addcontentsline{toc}{chapter}{Introduction}

Le projet de développement informatique nous a offert l'opportunité de créer le programme que nous voulions.
Divers sujets se présentaient à nous. Nous étions déterminés à trouver un projet nous permettant de mettre en œuvre plusieurs domaines de l'informatique.
Nous avons choisi le Rubik's Cube, un casse-tête fascinant: il s'agit d'un cube constitué de 3x3x3 sous-cubes colorés pouvant tourner selon trois axes.
Le Rubik's Cube se distingue par l'immense complexité de sa résolution contrastant avec sa simplicité apparente.
Chaque personne capable de résoudre le cube doit suivre une méthodologie stricte et faire preuve d'une certaine intuition.
L'objectif derrière le choix de ce célèbre casse-tête en tant que sujet est d'apprendre des domaines de l'informatique qui sont nouveau pour nous.

Le défi que nous nous sommes lancés est d'apprendre à une machine à résoudre ce casse-tête.
Pour cela nous devrons nous confronter à plusieurs problématiques.
La première, la plus évidente, est sa résolution. Nous allons avoir besoin de faire appel à des connaissances mathématiques et algorithmiques\cite{cite1}
De plus nous voulons donner un aspect pratique à notre projet. Ainsi, l'utilisateur devra pouvoir réussir à résoudre son cube grâce au programme.
Il est donc indispensable d'élaborer une interface utilisateur.
Nous devrons aussi ajouter une solution simple pour reconnaître la configuration du cube dans le programme. Nous utiliserons un algorithme de \textit{computer vision} sur un flux vidéo acquis en temps réel par une webcam.
Ainsi, ce projet regroupe les thématiques du traitement d'image, de l'algorithmie et du traitement d'image.


 
\chapter{Cahier des charges}
\section{Premier prototype}
Il s'agit premièrement d'implémenter la fonction basique de notre programme: la résolution du cube. Nous allons donc lui fournir en entrée un fichier contenant les couleurs de toutes les facettes du cube. Le programme cherchera à le résoudre et renverra les mouvements nécessaires à sa résolution. Par exemple en utilisant la notation de Singmaster\cite{cite11}.
Différentes méthodes sont pour l'instant envisagées:
\begin{itemize}
    \item La première consiste à implémenter un algorithme de résolution simple: par exemple le premier algorithme généralement appris par les humains.\cite{cite2} Le principe de celui-ci est d'envisager la résolution dans l'ordre suivant: d'abord une face, puis la première couronne, la seconde couronne et enfin la dernière face\cite{cite2}. Pour résoudre le Rubik's Cube de cette manière, le programme devra être capable de reconnaître la configuration dans laquelle se trouve le cube à un moment donné et lui
    appliquer une procédure, c'est-à-dire une suite prédéfinie de rotations. Les autres méthodes utilisées par les humains reposent sur ce même principe avec différentes configurations et procédures. Dans le cas des méthodes dites avancées, c'est-à-dire permettant de résoudre le Rubik's Cube en un plus petit nombre de rotations, le nombre de configurations et de procédures à apprendre augmente drastiquement.
\item Un algorithme reposant sur une machine de Boltzmann\cite{cite10} est également envisagée. Cet algorithme aurait une fonction d'évaluation de l'énergie du cube (plus le cube est proche de sa résolution, plus son énergie est faible) et chercherait à minimiser cette énergie via des rotations aléatoires ou des procédures aléatoires. Ces procédures auraient d'autant plus de chances d'être choisie que leur action réduit l'énergie du cube. Cet algorithme pourrait être utilisé conjointement avec le premier pour la résolution.
\item Un troisième algorithme dit en deux phases consiste à utiliser un algorithme A*\cite{cite3} (parcours d'un arbre en profondeur amélioré par une heuristique) pour placer les angles du cube au bon endroit lors de la première phase. La seconde phase consiste à utiliser réduire le nombre des rotations possibles à un groupe de cardinal plus petit.\footnote{Plus d'information sur les groupes du cube \cite{cite12}} Cet algorithme permet de résoudre le Rubik's Cube en une vingtaine de coups, mais est également plus dur à implémenter\cite{cite0}. Cet
    algorithme est une amélioration d'un algorithme passant par 5 groupes\cite{cite4}.
\end{itemize}

\section{L'affichage en 3D}
La prochaine étape du développement consiste à faire une interface homme machine. Il faut que l'utilisateur puisse visualiser les étapes successives de résolution du Rubik's Cube. Pour cela nous comptons afficher une fenêtre avec un bouton pour aller à l'étape suivante et une image en 3D de la configuration du cube que l'utilisateur devrait avoir dans les mains.

\section{\textit{Computer vision}}
Une fois que l'affichage a été réalisé, il faut pouvoir permettre de rentrer la configuration du cube au départ. Pour cela nous présenterons les différentes faces du cube à la caméra et un algorithme de traitement d'image\footnote{L'algorithme pourra être réalisé grâce à la bibliothèque OpenCV\cite{cite7}} sera capable de reconstituer le cube virtuellement.




\chapter{Structure globale}


\section{Le projet en général}

\begin{figure}[h]
\begin{center}
	\makebox[\textwidth]{\includegraphics[width=.8\paperwidth]{diagrammes/projet.png}}
\end{center}
\caption{GUI}
\end{figure}

Nous avons décidé de découper notre projet en plusieurs classes.
La classe principale de notre application est \textit{RubikInt}.
Celle-ci possède trois classes: l'interface graphique (ici noté \textit{GUI}), l'objet stockant la configuration du RubiksCube et celui en charge de charger la configuration et créer une instance de \textit{RubiksCube}.
La classe principale utilise également un résolveur qui recherchera les mouvements pour résoudre le cube.

\section{L'object Rubik's Cube}

\begin{figure}[h]
\begin{center}
	\makebox[\textwidth]{\includegraphics[width=.8\paperwidth]{diagrammes/rubikscube.png}}
\end{center}
\caption{GUI}
\end{figure}

Nous avons créé une classe abstraite pour permettre de multiples implémentations du Rubik's.
Un RubiksCube est "\textit{Rotatable}" et "\textit{Renderable}", il hérite donc de ces interfaces qui définissent son comportement.



\chapter{Interface}
\section{Affichage 3D}
Nos recherches sur la façon de réaliser un rendu 3D en Java nous a amené à utiliser une bibliothèque : \textit{Java3D}\cite{cite5}.
Même si le rendu à un instant t est satisfaisant (voir image ci-dessous), il ne nous permet pas, du moins de façon optimisée, de réaliser des animations comme la rotation du cube, le rendu étant figé une fois affiché.
\begin{figure}[h]
\begin{center}
	\makebox[\textwidth]{\includegraphics[width=.4\paperwidth]{diagrammes/rendu3D.png}}
\end{center}
\caption{GUI}
\end{figure}
Cet essai nous a permis de mieux cerner notre besoin. 
Il est nécessaire d'avoir un rendu 3D dans lequel l'objet affiché reste modifiable, dans notre cas de pouvoir réaliser les animation de rotation du cube. C'est pourquoi nous avons finalement décidé de réaliser tout cela à l'aide de la bibliothèque : \textit{OpenGL}\cite{cite6}.

\section{Interface fenêtrée}
Cet affichage 3D devra être intégré dans une interface, il sera accompagné de boutons pour rendre notre programme de résolution autonome et facile d'utilisation. Nous pouvons voir une ébauche d'utilisation de la bibliothèque \textit{Swing}\cite{cite9} sur la photo ci-dessus. L'objectif de cette première fenêtre est de faire tourner la face du cube de la couleur associée au bouton lorsque l'utilisateur clique dessus.

À terme, l'interface sera scindée en trois parties selon la logique suivante :

\begin{itemize}
    \item Un accueil pour que l'utilisateur puisse choisir la façon dont il veut résoudre son Rubik's cube (avec ou sans prise de photo d'un vrai cube)
    \item Une interface permettant la prise de photo du cube et donc de récupérer sa configuration 
    \item Une interface pour la résolution du cube.
\end{itemize}
Notre objectif est de rendre l'interface le plus \textit{user-friendly} possible.




\chapter{Résoudre un Rubik's cube}

\section{Le besoin}
Cette partie est le coeur du projet: il s'agit de créer un algorithme étant capable d'à partir d'un Rubik's Cube quelconque, de trouver la suite des rotations pour le résoudre. Celui-ci doit être le plus rapide possible, tant en terme de temps d'exécution qu'en terme de complexité de la solution trouvée, c'est à dire le nombre de rotations qui seront effectuées pour résoudre le cube.

\section{Modéliser un Rubik's Cube}

Dans ce projet, nous nous sommes rendu compte que modéliser un Rubik's Cube était moins trivial qu'il n'en avait l'air.
Le modèle qu'on cherchait devait respecter plusieurs critères. 
\begin{enumerate}
    \item Premièrement nous devions avoir une interface pour récupérer la couleur de chaque facette à tout moment.
    \item Ensuite il fallait qu'on ait une fonction qui simule une rotation sur une couronne quelconque.
    \item Enfin, notre modélisation devait être assez efficace pour permettre une résolution qui ne prenne pas trop de temps.
\end{enumerate}
\subsection{Les premières idées naïves}

\subsubsection{Le tableau de facettes}
La première idée qui nous est venue est de stocker la couleur de chaque facette dans un tableau.
Cette méthode n'est pas très efficace: créer une fonction pour simuler la rotation d'une couronne s'avère être compliqué.
De plus la complexité en mémoire n'est pas très bonne.

\subsubsection{La liste chaînée}
Similairement à cette première méthode, nous avons également songé à stocker les couleurs dans une liste chaînée en 2 dimensions.
On aurait "mis à plat" le Rubik's Cube et on aurait parcouru ses facettes sur un plan infini.

\subsubsection{Les cubies}
La deuxième idée que nous avons eu était de considérer les \textit{cubies}, petits cube constituant le Rubik's Cube en 3x3x3.
La rotation semblait être plus facile car nous aurions pu travailler avec des transformations d'angles et de position dans l'espace.
Cependant cette méthode faisait apparaître des complications d'implémentations rendant notre code très lourd.

\subsection{Le tableau de permutations}
Nous avons finalement décidé d'adopter une approche plus mathématique : le tableau de permutations.

Cela consiste en un tableau de 6*8=48 cases (six faces et 8 facettes qui peuvent bouger).
Le Rubik's résolu est alors représenté par un tableau rempli de 0 à 47 (dans l'ordre).
Effectuer une rotation est alors équivalent à appliquer une permutation sur le tableau.
Cela s'implémente simplement et l'algorithme est efficace.
Retrouver la couleur en fonction du numéro dans la case du tableau est aisé en définissant les permutations.

\begin{figure}[h]
\begin{center}
    \makebox[\textwidth]{\includegraphics[width=.5\paperwidth]{diagrammes/perm.png}}
\end{center}
    \caption{Les permutations du Rubik's}
\end{figure}

C'est finalement cette approche que nous avons décidé de garder car elle nous permettait de garder un code simple et efficace.

\section{Algorithme de résolution}
Une fois que nous avons créé un objet représentant le cube que nous pouvions manipuler virtuellement, nous nous sommes attaqué à la partie résolution.
Résoudre un Rubik's Cube nécessite d'être rigoureux et surtout méthodique.
C'est donc pourquoi nous avions initialement pensé qu'un ordinateur serait parfaitement adapté pour appliquer une méthode de résolution.
Cependant, nous nous sommes aperçu que il y a toujours une partie d'intuition dans les méthodes de résolution (par exemple faire la première face).
Cela nous a donc posé une difficulté.

\subsection{Les premiers essais}
Le premier essai a été d'adapter la méthode de résolution qu'un humain met en place lorsqu'il apprend à résoudre le cube. Cette méthode se repose sur des manoeuvres connues qui doivent être exécutées dans un certain ordre prédéterminé.
Ces manoeuvres doivent êtres exécutées lorsqu'on détecte une certaine configuration, c'est à dire lorsque la configuration du cube correspond à l'une des configurations dans laquelle la manoeuvre peut être exécutée. Cette approche devait par ailleurs être couplée avec le système de validateur pour permettre de couvrir plus de cas particuliers (dans lesquels il faut par exemple exécuter une même manoeuvre plusieurs fois, pour retrouver des configurations détectables).
Cette méthode a été finalement abandonnée pour plusieurs raisons: Tout d'abord elle est particulièrement rébarbative à implémenter et à debugger, car les erreurs dans les manoeuvres, et surtout dans les configurations sont particulièrement difficiles à détecter.
De plus, cette approche présentait un gros défaut qui apparaissait lors de la résolution de la première croix blanche sur le Rubik's Cube, qui est faite de manière très instinctive par l'homme, et qui nécessitait une multiplication trop importante du nombre de manoeuvres prédéterminées.
L'approche à donc été abandonnée.
\subsection{La solution retenue}
Nous avons finalement décidé de simplier au maximum de résolution.
La solution retenue a donc été basée sur de l'exploration de graphe en largeur et sur système de validateur.

\subsubsection{Les validateurs}
Les validateurs sont des éléments importants dans notre système de résolution.
Ils prennent en argument une configuration de Rubik's Cube et renvoient vrai ou faux.
Ils permettent de déterminer si une étape de résolution est finie.
Par exemple, un validateur pourra être chargé de déterminer si la croix blanche est faite, ou alors regarder si la deuxième couronne est faite.

Ainsi, tout d'abord nous lançons notre algorithme de résolution avec le premier validateur (celui qui vérifie la croix blanche).
Cet algorithme reverra une configuration de Rubik's valide (donc avec la croix blanche de faite).

Ensuite nous relançons l'algorithme avec le validateur 1 + le validateur 2 (qui vérifiera que les coins de la face blanche sont faits.
On obtiendra donc un cube qui est valide selon les deux premiers validateurs: la face blanche sera faite.

On relance successivement la résolution avec de plus en plus de validateurs (et donc de contraintes) jusqu'à ce que le Rubik's soit totalement résolu.

\subsubsection{L'algorithme de résolution}
Comme décrit précédemment, nous avons implémenté un algorithme permettant de partir d'une configuration du cub et de trouver les mouvements nécessaires pour qu'il passe le validateur.

Pour cela nous avons utilisé un algorithme de parcours en profondeur.
Nous lui fournissons une liste de Manoeuvres (ensemble de rotations) qu'il peut faire, et il essaiera successivement toutes les manoeuvres jusqu'à trouver le bon RubiksCube.
Bien entendu, les manoeuvres fournies sont les manipulations décrites par les techniques de résolution.

En faisant cela, nous nous assurons de trouver la solution optimale entre chaque étape.

\subsubsection{La fin}
Une fois que nous avons une liste de rotations pour résoudre le Rubik's, nous appliquons un dernier algorithme.
Il faut vérifier qu'il n'y a pas de mouvements inutiles.
Par exemple: Tourner une couronne dans le sens trigonométrique puis horaire.
Nous supprimons donc ces mouvements inutiles en passant sur la liste des rotations plusieurs fois en détectant les mouvements contraires et en les supprimant.






\chapter{Computer vision}

\section{Le besoin}
L'utilisateur doit pouvoir rentrer la configuration de son cube dans le programme pour qu'il le résolve.
La façon la plus simple pour l'utilisateur de faire ça nous semblait être utiliser sa webcam.
Il aurait juste suffit à la personne de montrer son Rubik's à la webcam et le logiciel aurait "vu" les couleurs des différentes faces.

\section{Le choix de la bibliothèque}
L'idée au départ, était bien évidemment d'utiliser la bibliothèque \textit{OpenCV}, grande prêtresse dans le monde du traitement d'image.
Cependant, on s'est vite rendu compte que c'était un véritable calvaire d'intégrer ses packages nécessaire à son utilisation dans notre 
environnement de développement en Java. Des solutions existent comme le projet amateur JavaCV (cf . \url{https://github.com/bytedeco/javacv}).
Encore une fois, même si cela nous permettais d'avoir accès aux méthodes d' \textit{OpenCV}, il devient très délicat d'intégrer notre webcam
dans notre interface \textit{Swing} selon les versions utilisé d' textit{OpenCV}. Après des recherches laborieuse durant plusieurs semaine 
sur internet, notre choix s'est porté sur la bibliothèque amateur \textit{webcam-capture} (cf. \url{https://github.com/sarxos/webcam-capture}).

Malheureusement, celle bibliothèque ne propose aucune méthode d'analyse d'image, cette dernière est réalisée spécialement pour intégrer une webcam
sur une interface \textit{Swing} (quel aubaine !). Nous l'aurons donc, vous l'aurez compris, définir une méthode d'analyse de l'image.

\section{Analyse d'image}
Notre but est, je vous le rappel, de pouvoir analyser les couleurs d'un Rubik's Cube afin de le résoudre. Plusieurs problèmes majeur vont se poser à la capture.
Détaillons les ensemble afin de vous expliquer la façon choisie de s'en afranchir.
Partons du postulat qu'il est très facile de savoir la couleur d'un pixel sur une image.
En effet, il suffit d'appliquer la méthode \textit{getRGB} à une image de type \textit{BufferedImage}.
Supposons donc qu'il est facile de connaitre la couleur d'un pixel, faut-il encore s'assurer que le pixel en question est celui d'une face de notre Rubik's Cube.

Comme je vous l'ai dit, nous avons décidé de nous affranchir d'\textit{OpenCV}, il est donc impensable pour nous d'utiliser des méthodes de reconnaissance de contour
afin de connaitre la position sur l'image de notre facette. 
C'est pourquoi nous avons choisi d'imposer la position des facette à l'utilisateur en position, par dessus l'image sur l'interface (cf. l'image InteractivSolver\_capture
dans le chapitre 3), des points indiquant à l'utilisateur de placer les facettes par juxtaposition. De cette façon, qui est certe moins "user-friendly", nous avons simplement 
à analyser un nombre fixe de pixels. Encore mieux, ce sont toujours les même pixels qui seront analysé.

A partir d'ici, nous avons quasiment terminé. En effet, à une image donnée, on sait la position des facettes et il est possible de récupérer le code RGB de cette dernière. 
En pratique, nous prenont non pas 1 pixel, mais 25 (5*5) puis nous moyennons leur valeur pour s'éviter des valeurs abérante.

Un nouveau problème est alors apparut : la dépendance du code RGB à l'environnement de capture. Le code RGB ne prenant pas en compte la luminosité ou la saturation, ses valeurs 
sont très fluctuante selon l'endroit où nous capturons notre Rubik's Cube. Il a alors fallut trouver un autre codage couleur : Le HSV.
Le HSV est l'acronyme de Hue, Saturation, Value of Brightness (parfois appelé HSB). Ce code, définie par trois nombre compris entre 0 et 1, permettent de définir une couleur de façon 
beaucoup plus performante dans différents environnement. La valeur de H, qui définie la couleur en elle même, est beaucoup plus stable grâce aux valeurs de S et V. Il est alors beaucoup 
plus facile de définir des seuils empirique pour choisir les couleurs.

Ce sont ces seuils qui ont été défini par la méthode \textit{defineColor} dans \textit{InteractivSolver\_capture}.
Ces seuils sont, comme je viens de le mentionner, définie empiriquement par échantillonnage (ci-dessous des valeurs moyenné de HSV des 6 couleurs faites à partir de 150 mesures par couleurs dans différents environnement).

\begin{figure}[H]
\begin{center}
	\makebox[\textwidth]{\includegraphics[width=.8\paperwidth]{diagrammes/Echantillonnage.png}}
\end{center}
	\caption{ \textit{Echantillonnage des valeurs HSV}}
\end{figure}

Ainsi, on peut remarquer que toute les couleurs on un domaine de H bien distinct mise à part la couleur blanche et bleu pour qui il faut prendre la valeur de S qui sont bien différents entre les deux pour pouvoir les déterminer.

En pratique, ce système fonctionne dans la grande majorité des cas ce qui en fait un choix concluant pour la suite. Les analyses de couleurs fiable quelque soit la luminosité ambiante est un problème complexe qui fait l'objet de recherches
approfondie donc on peut s'estimer heureux d'avoir un système aussi fiable.

Maintenant, nous pouvons, à partir d'une image, analyser les facettes et connaitre leur couleur. Nous venons donc d'élaborer un système de reconnaissance des couleurs.

Le dilemme maintenant est de s'assurer de la positions des différentes facette que présente l'utilisateur pour éviter les configurations éronnées (comme un coin Blanc-Jaune-Blanc par exemple).
Nous demandons donc (cf. \textit{InteractivSolver\_capture} pour voir le code donnant les instructions) à l'utilisateur de poser face caméra le cube dans une position particulière (typiquement en fixant la face au dessus de celle présentée) de sorte 
à connaitre systématiquement les coordonnées des différentes facettes.

Nous y voilà, nous avons un système de Computer Vision qui est maintenant fonctionnel du Rubik's Cube, la grande diffculté qui reste étant la conversion de notre liste de couleurs que nous obtenons en une liste de permutations pour définir le Rubiks's Cube
et pourvoir le résoudre.








\chapter*{Conclusion}

Pour conclure, nous pouvons affirmer sans hésitation que ce projet nous a beaucoup appris.
Tout d'abord, nous avons du nous poser des questions de conception sur l'ensemble du projet et des différents éléments qui interagissent entre eux.
Pour se faire nous avons du apprendre à nous organiser en équipe, à faire un planning et une répartition équitable des tâches entre les membres. 
La conctrainte du code nous a poussé à uitliser utiliser des outils de versionning (tel que git) et à aquérir les bonnes pratiques associées.

Puis, à la phase d'implémentation, nous nous sommes confrontés à des problèmes qui nous semblaient simples en apparence mais qui se sont avérés compliqués algorithmiquement parlant.
Cela nous a permis d'approfondir nos connaissances sur le langage de programmation Java en utilisant des bibliothèques que nous n'avions jamais manipulé.
Nous avons été sensibilisés à la partie plus théorique de la programmation : nous avons du réfléchir sur le coût en temps de nombreuses fonctions.


Enfin, ce projet est venu enrichir notre expérience de travail en équipe.

Aujourd'hui nous sommes fiers de vous présenter notre programme, Rubik'INT. Nous espérons qu'il pourra servir et aider des personnes à apprendre à jouer au Rubik's cube.


\newpage

%récupérer les citation avec "/footnotemark"
\nocite{*}

%choix du style de la biblio
\bibliographystyle{plain}
%inclusion de la biblio
\bibliography{bibliographie.bib}
%voir wiki pour plus d'information sur la syntaxe des entrées d'une bibliographie

\end{document}
